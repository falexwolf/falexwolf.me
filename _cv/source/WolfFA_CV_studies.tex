%\documentclass[10pt]{scrartcl}
%\usepackage{graphicx}
%\usepackage{array}
%\usepackage{setspace}
%\setkomafont{sectioning}{\rmfamily\bfseries}
%\usepackage[left=2cm, right=2cm, top=2cm, bottom=2cm]{geometry}

 %ohne kopfzeile top=2.5cm
\renewcommand{\widthA}{0.7\textwidth}
\renewcommand{\widthB}{0.13\textwidth}
\renewcommand{\widthC}{0.1\textwidth}


%\begin{document}

\hypertarget{courses}{}
\subsection*{Courses during studies}
Coming from an interest in the arts and the social sciences, in the first semesters, it took me a while to adapt to the requirements of maths and physics. In summer 2006, U Augsburg was ranked among the top five universities for studying physics in Germany, with higher rankings in the category ``teaching'' than the top-ranked TU Munich and LMU Munich. During bachelor studies, we had 5 fellow students with scholarships from \textit{Studienstiftung} or \textit{Max Weber Programm}.

Notes on reading the table:
All ECTS marked with an ``$^*$'' are \textit{additional} courses that were not required for the BSc and MSc degrees. For these courses, exams have been taken just for fun' without preparation. 
``P'' stands for ``participation''%
\footnote[3]
{
Ungraded talk in a seminar or simple participation in a lecture -- in case of employment, validation of this work as practical.
}. 

\onehalfspacing{
\noindent \begin{longtable}[t]{p{0.1\textwidth} p{0.86\textwidth}}
\textbf{\normalsize{Summary}}
\\ \hline \hline
Grades: 	&BSc in Physics 1.50 and MSc in Physics 1.05 \textit{\footnotesize$\vert$ top of class for both degrees} \\
ECTS: 	&400 of required 300 ECTS \textit{\footnotesize$\vert$ of which BSc 180 + MSc 120 + additional 100}  \\
		& \textit{\footnotesize at the faculties Phys 294 + Math 62 +	 Phil 24 +	 Comp Sc 22}
\\ \hline \hline
\end{longtable}

\footnotetext[4]{~All grades of this semester on a scale 0 (worst) to 20 (best), exams taken ``just for fun'' without preparation.}

\noindent \begin{longtable}[t]{p{0.98\textwidth}}
\textbf{\normalsize{Master of Science in Physics}}
Grade 1.05, ECTS 161 of which MSc 97 + additional 64
\\ \hline \hline
\end{longtable}
\vspace{-2em}\noindent \begin{longtable}[t]{p{\widthA} p{\widthB} p{\widthC}}
Course name
& Grade
& ECTS \\ \hline
\textbf{Summer term 2011} (10th semester, ECTS 34 / MSc 34)\\
Master's thesis
& 1.0
& 26\\
Master's thesis defense
& 1.7
& 4\\
Theory of interacting electrons (U Eckern, Phys)
& P
& 4\\
 \hline
\textbf{Winter term 2010} (9th semester, ECTS 40 / add 40)\\
\textit{within the ``Parcours Th\'eorique'' -- 2nd year theortical physics master program at \textbf{ENS Paris}} \textit{-- these are all additional coures, translated French titles}\\
Computational physics (W Krauth, Phys)
& 18.0\footnotemark[4]
& 8$^*$\\
Statistical physics and applications (H Hilhorst, Phys)
& 12.5
& 8$^*$\\
Statistical field theory (F David, J Jespersen, Phys)
& 11.0
& 8$^*$\\
Group theory and symmetries (J-B Zuber, Phys)
& 8.25
& 8$^*$\\
Quantum field theory (A Bilal, Phys)
& 10.3 
& 8$^*$\\
\hline
\textbf{Summer term 2010} (8th semester, ECTS: 51 / MSc 27 + add 24)\\
\textit{02.10 - 06.10: Student researcher, \textbf{Georgetown U}, validated as practical }
& P
& 15 \\
Statistical physics far from equilibrium II  (M Kollar, Phys)
& 1.0
& 4\\
Theory of magnetism (T Kopp, Phys)
& 1.0
& 8\\
\textit{Additional courses}\\
Complex analysis (Wendland, Math)
& 1.3
& 8$^*$\\
Linear Algebra II (Ulm, Math)
& 3.3
& 8$^*$\\
Linear Algebra I (Hackenberger, Math)
& 4.0
& 8$^*$\\
 \hline
\textbf{Winter term 2009} (7th semester, ECTS 36 / MSc 36)\\
\textit{10.09 - 01.10: Tutor ``Statistical physics and Thermodynamics''}\\
Relativistic quantum field theory (T Kopp, Phys)
& 1.0
& 8\\
Theoretical condensed matter physics (D Vollhardt, Phys)
& 1.0
& 8\\
Ordinary differential equations (F Colonius, Math)
& 1.0
& 8\\
Theory of phase transitions (K Ziegler, Phys)
& 1.0
& 8\\
Statistical physics far from equilibrium I (E Lutz, Phys)
& 1.0
& 4\\
\hline \hline
\end{longtable}

\noindent \begin{longtable}[t]{p{0.98\textwidth}}
\textbf{\normalsize{Bachelor of Science in Physics}}
Grade 1.50, ECTS 239 of which BSc 180 + MSc 23 + additional 36
\\ \hline \hline
\end{longtable}
\vspace{-2em}\noindent \begin{longtable}[t]{p{\widthA} p{\widthB} p{\widthC}}
Course name
& Grade
& ECTS \\ \hline
\textbf{Summer term 2009} (6th semester, ECTS 63 / BSc 24 + MSc 23 + add 16) \\
\textit{07.09 - 10.09: Student researcher, U Augsburg, validated as practical }
& P
& 15 (MSc)\\
Bachelor Thesis (D Vollhardt and M Kollar, Phys)
& 1.0
& 12\\
Electrodynamics and classical field theory (P Haenggi, Phys)
& 1.7
& 8\\
Particular problems of quantum theory (U Eckern, Phys)
& P
& 4\\
\textit{Additional courses}\\
Numerics II (Hoppe, Math)
& 1.3
& 8 (MSc)\\
Theoretical condensed matter physics II (A Kampf, Phys)
& P
& 8$^*$\\
History of the philosophy of the present (C Schroer, Phil)
& 1.7
& 4$^*$\\ 
Current problems of many-body theory (D Vollhardt, Phys)
& P (no record)
& 4$^*$\\
\hline
\textbf{Winter term 2008} (5th semester, ECTS 40 / BSc 28 + add 12)\\
Statistical physics and Thermodynamics (D Vollhardt, Phys)
& 1.0
& 8\\
Applied optics (B Stritzker, Phys)
& 1.3
& 8\\
Advanced practical
& 1.0
& 12\\
\textit{Additional courses}\\
Wittgenstein: Philosophische Untersuchungen (Tatjevskaya, Phil)
& P
& 4$^*$\\
Introduction to Logic (Tatjevskaya, Phil)
& 1.0
& 4$^*$\\
History of the philosophy of the modern age (C Schroer, Phil)
& P (no record)
& 4$^*$\\ \hline
\textbf{Summer term 2008} (4th semester, ECTS 34 / BSc 34)\\
\textit{02.08 - 06.08: Student researcher, U Augsburg} \\
Quantum Mechanics (D Vollhardt, Phys)
& 1.3
& 8\\
Condensed matter physics (J Mannhart, Phys)
& 1.0
& 8\\
Numerics for physicists (Hoppe, Math)
& 1.0
& 6\\
Beginners practical II
& 1.0
& 8\\
Introduction to Latex (G Hammerl, Phys)
& P
& 4\\ \hline
\textbf{Winter term 2007} (3rd semester, ECTS 38 / BSc 30 + add 8)\\
Classical Mechanics (I Goychuk, Phys)
& 2.3
& 8\\
Atom and molecular physics (J Mannhart, Phys)
& 1.0
& 8\\
Design of electronic systems (S Uhrig, Comp)
& 1.0
& 6\\
Beginners practical I
& 1.3
& 8\\
\textit{Additional courses}\\
Introduction to philosophy (Hofweber, Phil)
& P
& 4$^*$\\
Introduction to philosophy of science (W Pietsch, Phil)
& P (no record)
& 4$^*$\\ \hline
\textbf{Summer term 2007} (2nd semester, ECTS: 32)\\
Mathematics for physicists II (G Ingold, Phys)
& 1.7
& 8\\
General physics II (F Haider, Phys)
& 2.0
& 8\\
Computer Science II (Kie\ss ling, Comp) \textit{\footnotesize$\vert$ ranked 3rd out of 200}
& 2.0
& 8\\
Analysis II (D Bl\"omker, Math)
& 2.0
& 8\\ \hline
\textbf{Winter term 2006} (1st semester, ECTS: 32)\\
Mathematics for physicists I (G Ingold, Phys)
& 1.0
& 8\\
General physics I (F Haider, Phys) \textit{\footnotesize$\vert$ ranked 3rd out of 160}
& 2.0
& 8\\
Computer Science I (Kie\ss ling, Comp) \textit{\footnotesize$\vert$ among top 10 out of 200}
& 3.0
& 8\\
Analysis I (D Bl\"omker, Math)
& 2.3
& 8\\ \hline \hline
\end{longtable}

%\end{document}


